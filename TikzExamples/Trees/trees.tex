%
% This is a borrowed LaTeX template file for lecture notes for CS267,
% Applications of Parallel Computing, UCBerkeley EECS Department.
% Now being used for CMU's 10725 Fall 2012 Optimization course
% taught by Geoff Gordon and Ryan Tibshirani.  When preparing 
% LaTeX notes for this class, please use this template.
%
% To familiarize yourself with this template, the body contains
% some examples of its use.  Look them over.  Then you can
% run LaTeX on this file.  After you have LaTeXed this file then
% you can look over the result either by printing it out with
% dvips or using xdvi. "pdflatex template.tex" should also work.
%

\documentclass[twoside]{article}
    \setlength{\oddsidemargin}{0.25 in}
    \setlength{\evensidemargin}{-0.25 in}
    \setlength{\topmargin}{-0.6 in}
    \setlength{\textwidth}{6.5 in}
    \setlength{\textheight}{8.5 in}
    \setlength{\headsep}{0.75 in}
    \setlength{\parindent}{0 in}
    \setlength{\parskip}{0.1 in}
    
    %
    % ADD PACKAGES here:
    %
    \usepackage[T1]{fontenc}                
    \usepackage[utf8]{inputenc}             
    \usepackage[english,italian]{babel}
    \usepackage{amsmath,amsfonts,graphicx}
    \usepackage{tikz}
    \usetikzlibrary{shapes,arrows,fit,calc,positioning,trees,shadows}
        \tikzset{box/.style={draw, diamond, thick, text centered, minimum height=0.5cm, minimum width=1cm}}
        \tikzset{line/.style={draw, thick, -latex'}}
        \tikzstyle{vertex}=[draw,fill=black!15,circle,minimum size=20pt,inner sep=0pt]
    \usepackage{pst-tree}
        \def\psedge{\ncangles[angleA=-90,angleB=90]}
        \psset{levelsep=14mm,treesep=1.8cm,nodesep=5pt}
    
    %
    % The following commands set up the lecnum (lecture number)
    % counter and make various numbering schemes work relative
    % to the lecture number.
    %
    \newcounter{lecnum}
    \renewcommand{\thepage}{\thelecnum-\arabic{page}}
    \renewcommand{\thesection}{\thelecnum.\arabic{section}}
    \renewcommand{\theequation}{\thelecnum.\arabic{equation}}
    \renewcommand{\thefigure}{\thelecnum.\arabic{figure}}
    \renewcommand{\thetable}{\thelecnum.\arabic{table}}
    
    %
    % The following macro is used to generate the header.
    %
    \newcommand{\lecture}[4]{
       \pagestyle{myheadings}
       \thispagestyle{plain}
       \newpage
       \setcounter{lecnum}{#1}
       \setcounter{page}{1}
       \noindent
       \begin{center}
       \framebox{
          \vbox{\vspace{2mm}
        \hbox to 6.28in { {\bf CSC3001: graph theory
        \hfill Fall 2020} }
           \vspace{4mm}
           \hbox to 6.28in { {\Large \hfill Lecture #1: #2  \hfill} }
           \vspace{2mm}
           \hbox to 6.28in { {\it Author: #3 \hfill Scribes: #4} }
          \vspace{2mm}}
       }
       \end{center}
       \markboth{Lecture #1: #2}{Lecture #1: #2}
    }
    %
    % Convention for citations is authors' initials followed by the year.
    % For example, to cite a paper by Leighton and Maggs you would type
    % \cite{LM89}, and to cite a paper by Strassen you would type \cite{S69}.
    % (To avoid bibliography problems, for now we redefine the \cite command.)
    % Also commands that create a suitable format for the reference list.
    \renewcommand{\cite}[1]{[#1]}
    \def\beginrefs{\begin{list}%
            {[\arabic{equation}]}{\usecounter{equation}
             \setlength{\leftmargin}{2.0truecm}\setlength{\labelsep}{0.4truecm}%
             \setlength{\labelwidth}{1.6truecm}}}
    \def\endrefs{\end{list}}
    \def\bibentry#1{\item[\hbox{[#1]}]}
    
    %Use this command for a figure; it puts a figure in wherever you want it.
    %usage: \fig{NUMBER}{SPACE-IN-INCHES}{CAPTION}
    \newcommand{\fig}[3]{
                \vspace{#2}
                \begin{center}
                Figure \thelecnum.#1:~#3
                \end{center}
        }
    % Use these for theorems, lemmas, proofs, etc.
    \newtheorem{theorem}{Theorem}[lecnum]
    \newtheorem{lemma}[theorem]{Lemma}
    \newtheorem{proposition}[theorem]{Proposition}
    \newtheorem{claim}[theorem]{Claim}
    \newtheorem{corollary}[theorem]{Corollary}
    \newtheorem{definition}[theorem]{Definition}
    \newenvironment{proof}{{\bf Proof:}}{\hfill\rule{2mm}{2mm}}
    \newtheorem{example}[theorem]{Example}

    % **** IF YOU WANT TO DEFINE ADDITIONAL MACROS FOR YOURSELF, PUT THEM HERE:
    
    \newcommand\E{\mathbb{E}}
    
    \begin{document}
    %FILL IN THE RIGHT INFO.
    %\lecture{**LECTURE-NUMBER**}{**DATE**}{**LECTURER**}{**SCRIBE**}
    \lecture{5}{Nov 13}{Griffith Ke}{Griffith Ke}
    %\footnotetext{These notes are partially based on those of Nigel Mansell.}
    
    \begin{definition}[Circuit-free Graph]
        A graph is said to be circuit-free iff it has no circuits.
    \end{definition}

    \begin{definition}[Tree]
        A graph is called a tree iff it is circuit-free and connected.
    \end{definition}

    \begin{definition}[Trivial Tree]
        A trivial tree is a graph that consists of a single vertex.
    \end{definition}

    \begin{definition}[Forest]
        A graph is called a forest if, and only if, it is circuit-free and not connected.
    \end{definition}
    
    \begin{example}[Taxonomy Tree]
        Taxonomy tree is a father set of decision tree.
        \begin{figure}[htbp]
            \hspace*{1.3cm}
            \pstree{\Tr[name=class]{$\gamma$-proteobacteria}}{%
            \pstree{\Tr[name=order]{Alteromonadales}}{%
                \pstree[thistreesep=3mm]{\Tr[name=family]{Alteromonadaceae}}{%
                \Tr[name=genus]{Glaciecola}
                \Tr{Alteromonas} 
                \Tr{Agarivorans}%
            }}
            \pstree{\Tr{Vibrionales}}{%
                \pstree{\Tr{Vibrionacae}}{%
                \Tr{Vibrio}
            }}
            }

            \rput[rc](0,0|class){Class}\rput[rc](0,0|order){Order}%
            \rput[rc](0,0|family){Family}\rput[rc](0,0|genus){Genus}%
            \caption{A taxonomy tree}
        \end{figure}
    \end{example}

    \begin{example}[Decision Tree]
        Nodes represent different decision point. Edges represent different decision results.
        \begin{figure}[htbp]
            \begin{tikzpicture}[edge from parent/.style={draw,-latex},
            level distance=2cm,
            level 1/.style={sibling distance=10cm},
            level 2/.style={sibling distance=3cm},scale=0.73]
            \tikzstyle{every node}=[rectangle,draw]    
            \node (Root) {$A < B$}
            child {
                node {$B < C$}    
                child { 
                    node {$A < B < C$} 
                    edge from parent node[left,draw=none] {yes}
                }
                child { 
                    node {$A < C$} 
                    child {
                        node {$A < C \leq B$}
                        edge from parent node[left,draw=none] {yes}
                    }
                    child {
                        node {$C \leq A < B$}
                        edge from parent node[right,draw=none] {no}
                    }
                    edge from parent node[right,draw=none] {no}
                }
                edge from parent node[left,draw=none] {yes $\;$}
            }
            child {
                node {$B < C$}
                child { 
                    node {$A < C$}     
                    child {
                        node {$B \leq A < C$}
                        edge from parent node[left,draw=none] {yes}
                    }   
                    child {
                        node {$B < C \leq A$}
                        edge from parent node[left,draw=none] {no}
                    }     
                    edge from parent node[left,draw=none] {yes}
                }
                child { 
                    node {$C \leq B \leq A$} 
                    edge from parent node[right,draw=none] {no}
                }
                edge from parent node[right,draw=none] { $\;$ no}
            };
            \end{tikzpicture}
            \caption{A decision tree for sorting three values.}
            \label{fig:sortingtree}
        \end{figure}
    \end{example}

    \begin{example}[Parse Tree]
        This work has proved useful in constructing compilers for high- level computer languages. In the study of grammars, trees are often used to show the derivation of grammatically correct sentences from certain basic rules. Such trees are called syntactic derivation trees or parse trees.
        \begin{tikzpicture}[font=\sffamily,thick,level/.style={sibling distance=50mm/#1}]
            \node[vertex] {Sentence}
                child {
                    node[vertex] {Noun phrase}
                    child {
                        node[vertex] {article}
                        child {
                            node[vertex] {the}
                        }
                    }
                    child {
                        node[vertex] {adjective}
                        child { node[vertex] {young} }
                    }
                    child {
                        node[vertex] {noun}
                        child { node[vertex] {man}}
                    }
                } child {
                    node[vertex] {Verb phrase}
                    child {
                        node[vertex] {verb}
                        child {
                            node[vertex] {caught}
                        }
                    }
                    child {
                        node[vertex] {noun phrase}
                        child {
                            node[vertex] {article}
                            child {
                                node[vertex] {the}
                            }
                        }
                        child {
                            node[vertex] {noun}
                            child {
                                node[vertex] {ball}
                            }
                        }
                    }
                }
            ;
        \end{tikzpicture}
    \end{example}
\end{document}
