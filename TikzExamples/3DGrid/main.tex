\documentclass[12pt]{article}
\usepackage[margin=1in]{geometry}
\usepackage[all]{xy}
\usepackage{amsmath,amsthm,amssymb,color,latexsym}
\usepackage{geometry}
\geometry{letterpaper}
\usepackage{graphicx}
\usepackage{tikz}
\usepackage{tabularx}
\usetikzlibrary[topaths,arrows,positioning]

\begin{document}

% transforms all coordinates the same way when used (use it within a scope!)
% (rotation is not 45 degress to avoid overlapping edges)
% Input: point of origins x and y coordinate
\newcommand{\myGlobalTransformation}[2]
{
    \pgftransformcm{1}{0}{0.4}{0.5}{\pgfpoint{#1cm}{#2cm}}
}

% draw a 4x4 helper grid in 3D
% Input: point of origins x and y coordinate and additional drawing-parameters
\newcommand{\gridThreeD}[3]
{
    \begin{scope}
        \myGlobalTransformation{#1}{#2};
        \draw [#3,step=2cm] grid (6,6);
    \end{scope}
}

\tikzstyle myBG=[line width=3pt,opacity=1.0]

% draws lines with white background to show which lines are closer to the
% viewer (Hint: draw from bottom up and from back to front)
%Input: start and end point
\newcommand{\drawLinewithBG}[2]
{
    \draw[->,white,myBG]  (#1) -- (#2);
    \draw[->,black,very thick] (#1) -- (#2);
}

% draws all horizontal graph lines within grid
\newcommand{\graphLinesHorizontal}
{
    \drawLinewithBG{1,1}{5,1};
    \drawLinewithBG{1,3}{5,3};
    \drawLinewithBG{1,5}{5,5};
}

% draws all vertical graph lines within grid
\newcommand{\graphLinesVertical}
{
    %swaps x and y coordinate (hence vertical lines):
    \pgftransformcm{0}{1}{1}{0}{\pgfpoint{0cm}{0cm}}
    \graphLinesHorizontal;
}

%draws nodes of the grid
%Input: point of origins x and y coordinate
\newcommand{\graphThreeDnodes}[2]
{
    \begin{scope}
        \myGlobalTransformation{#1}{#2};
        \foreach \x in {1,3,5} {
            \foreach \y in {1,3,5} {
                \node at (\x,\y) [circle] {};
                %this way circle of nodes will not be transformed
            }
        }
    \end{scope}
}

\begin{figure}
\centering
\begin{tikzpicture}

    %draws helper-grid:
    \gridThreeD{0}{0}{black!50};
    \gridThreeD{0}{2.125}{black!50};
    \gridThreeD{0}{4.25}{black!50};

    %draws lower graph lines and those in z-direction:
    \begin{scope}
        \myGlobalTransformation{0}{0};
        % \graphLinesHorizontal;
        \drawLinewithBG{1,1}{5,1}
        \drawLinewithBG{5,1}{5,3}
        \drawLinewithBG{5,3}{1,3}
        \drawLinewithBG{1,3}{1,5}
        \drawLinewithBG{1,5}{5,5}
        %draws all graph lines in z-direction (reset transformation first!):
        \node (thisNode) at (5,5) {};
        {
            \pgftransformreset
            \draw[->,white,myBG]  (thisNode) -- ++(0,2.125);
            \draw[->,black,very thick] (thisNode) -- ++(0,2.125);
        }
        \node (thisNode) at (1,1) {};
        {
            \pgftransformreset
            \draw[<-,white,myBG]  (thisNode) -- ++(0,4.25);
            \draw[<-,black,very thick] (thisNode) -- ++(0,4.25);
        }
    \end{scope}

    \begin{scope}
        \myGlobalTransformation{0}{2.125};
        \drawLinewithBG{5,5}{1,5}
        \drawLinewithBG{1,5}{1,3}
        \drawLinewithBG{1,3}{5,3}
        \drawLinewithBG{5,3}{5,1}
        \drawLinewithBG{5,1}{3,1}
        \node (thisNode) at (3,1) {};
        {
            \pgftransformreset
            \draw[->,white,myBG]  (thisNode) -- ++(0,2.125);
            \draw[->,black,very thick] (thisNode) -- ++(0,2.125);
        }
    \end{scope}

    \begin{scope}
        \myGlobalTransformation{0}{4.25};
        \drawLinewithBG{3,1}{3,3}
        \drawLinewithBG{3,3}{1,3}
        \drawLinewithBG{1,3}{1,5}
        \drawLinewithBG{1,5}{5,5}
        \drawLinewithBG{5,5}{5,1}
    \end{scope}

    % draws all graph nodes:
    \graphThreeDnodes{0}{0};
    \graphThreeDnodes{0}{2.125};
    \graphThreeDnodes{0}{4.25};

\end{tikzpicture}
\caption{n=3 path in 3D} \label{fig:3DIllus}
\end{figure}

\end{document}