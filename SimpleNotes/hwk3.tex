\documentclass[12pt]{article}

\usepackage[margin=1in]{geometry} 
\usepackage{amsmath,amsthm,amssymb,scrextend}
\usepackage{fancyhdr}
\usepackage{tikz}
\usepackage{array}
\usepackage{chngcntr}
\usetikzlibrary{decorations.pathreplacing}
\pagestyle{fancy}

% reset equation counter at each section
\counterwithin*{equation}{section}

\newcommand{\N}{\mathbb{N}}
\newcommand{\Z}{\mathbb{Z}}
\newcommand{\I}{\mathbb{I}}
\newcommand{\R}{\mathbb{R}}
\newcommand{\Q}{\mathbb{Q}}
\renewcommand{\qed}{\hfill$\blacksquare$}
\let\newproof\proof
\renewenvironment{proof}{\begin{addmargin}[1em]{0em}\begin{newproof}}{\end{newproof}\end{addmargin}\qed}
% \newcommand{\expl}[1]{\text{\hfill[#1]}$}

\newenvironment{motivation}[2][Motivation]{\begin{trivlist}
\item[\hskip \labelsep {\bfseries #1}\hskip \labelsep {\bfseries #2.}]}{\end{trivlist}}
\newenvironment{definition}[2][Definition]{\begin{trivlist}
\item[\hskip \labelsep {\bfseries #1}\hskip \labelsep {\bfseries #2.}]}{\end{trivlist}}
\newenvironment{theorem}[2][Theorem]{\begin{trivlist}
\item[\hskip \labelsep {\bfseries #1}\hskip \labelsep {\bfseries #2.}]}{\end{trivlist}}
\newenvironment{lemma}[2][Lemma]{\begin{trivlist}
\item[\hskip \labelsep {\bfseries #1}\hskip \labelsep {\bfseries #2.}]}{\end{trivlist}}
\newenvironment{problem}[2][Problem]{\begin{trivlist}
\item[\hskip \labelsep {\bfseries #1}\hskip \labelsep {\bfseries #2.}]}{\end{trivlist}}
\newenvironment{exercise}[2][Exercise]{\begin{trivlist}
\item[\hskip \labelsep {\bfseries #1}\hskip \labelsep {\bfseries #2.}]}{\end{trivlist}}
\newenvironment{example}[2][Example]{\begin{trivlist}
\item[\hskip \labelsep {\bfseries #1}\hskip \labelsep {\bfseries #2.}]}{\end{trivlist}} 
\newenvironment{reflection}[2][Reflection]{\begin{trivlist}
\item[\hskip \labelsep {\bfseries #1}\hskip \labelsep {\bfseries #2.}]}{\end{trivlist}}
\newenvironment{proposition}[2][Proposition]{\begin{trivlist}
\item[\hskip \labelsep {\bfseries #1}\hskip \labelsep {\bfseries #2.}]}{\end{trivlist}}
\newenvironment{corollary}[2][Corollary]{\begin{trivlist}
\item[\hskip \labelsep {\bfseries #1}\hskip \labelsep {\bfseries #2.}]}{\end{trivlist}}
    
\begin{document}
    
% --------------------------------------------------------------
%                         Start here
% --------------------------------------------------------------

\lhead{MAT3040 Homework 3}
\chead{Aron Ke}
\rhead{\today}

\section{Question 1}

\begin{proof}{}
    \noindent

    Prove this by contradiction.

    Suppose $\{T\mathbf{v}_1,\dots,T\mathbf{v}_k\}$ is linearly dependent in $W$. That is, $\exists \alpha_1,\dots,\alpha_k$ not all zero such that:

    \[
        \sum_{i=1}^{k}\alpha_iT\mathbf{v}_i=\mathbf{0}_W
    \]

    Then

    \[
        T\left(\sum_{i=1}^{k}\alpha_i\mathbf{v}_i\right)=\mathbf{0}_W
    \]

    So $\sum_{i=1}^{k}\alpha_i\mathbf{v}_i\in \textrm{Ker}(T)$

    $\because \{\mathbf{v}_1,\dots,\mathbf{v}_k\}$ are linearly independent

    $\therefore$

    \[
        \sum_{i=1}^{k}\alpha_i\mathbf{v}_i\not=\mathbf{0}_V
    \]

    However, $\because T$ is injective. That means, $\textrm{Ker}(T)=\{\mathbf{0}_V\}$.
    
    $\therefore$Contradiction. $\alpha_1,\dots,\alpha_k$ do not exist.

    Hence, $\{T\mathbf{v}_1,\dots,T\mathbf{v}_k\}$ is linearly independent in $W$.
\end{proof}

\pagebreak

\section{Question 2}

\begin{proof}{}
    \noindent

    $\forall \mathbf{w}\in W$

    $\because T$ is surjective.

    $\therefore\exists\mathbf{v}\in V$ such that

    \begin{equation}
        \mathbf{w}=T(\mathbf{v})
    \end{equation}

    $\because \{\mathbf{v}_1,\dots,\mathbf{v}_k\}$ spans $V$.

    $\therefore\exists\alpha_1,\dots,\alpha_k$ such that

    \begin{equation}
        \mathbf{v}=\sum_{i=1}^{k}\alpha_i\mathbf{v}_i
    \end{equation}

    By $(1)$ and $(2)$,

    \begin{equation}
        \begin{aligned}
            \mathbf{w}&=T(\mathbf{v})\\
            &=T(\sum_{i=1}^{k}\alpha_i\mathbf{v}_i)\\
            &=\sum_{i=1}^{k}\alpha_iT(\mathbf{v}_i)
        \end{aligned}
    \end{equation}

    Hence, $\{T\mathbf{v}_1,\dots,T\mathbf{v}_k\}$ spans $W$.
\end{proof}

\pagebreak

\section{Question 3}

\begin{proof}{}
    \noindent

    $\because T:V\rightarrow W$ is an isomorphism.

    $\therefore T$ is bijective. Then the following three conditions are satisfied.

    \begin{equation}
        \forall \mathbf{v}_1\in V,\mathbf{v}_2\in V,\mathbf{v}_1\not=\mathbf{v}_2\Longrightarrow T(\mathbf{v}_1)\not=T(\mathbf{v}_2)
    \end{equation}

    \begin{equation}
        \forall \mathbf{w}\in W,\exists\mathbf{v}\in V,\textrm{such that}\ \mathbf{w}=T(\mathbf{v})
    \end{equation}

    \begin{equation}
        \textrm{Ker}(T)=\{\mathbf{0}_V\}
    \end{equation}

    $\because Span(\{\mathbf{v}_1,\dots,\mathbf{v}_n\})=V$

    $\therefore$ By (2), $\forall\mathbf{w}\in W,\exists\alpha_1,\dots,\alpha_n,\mathbf{w}=T(\sum_{i=1}^{n}\alpha_i\mathbf{v}_i)=\sum_{i=1}^{n}\alpha_iT(\mathbf{v}_i)$. So

    \begin{equation}
        Span(\{T(\mathbf{v}_1),\dots,T(\mathbf{v}_n)\})=W
    \end{equation}

    Suppose $\{T(\mathbf{v}_1),\dots,T(\mathbf{v}_n)\}$ is linearly dependent.

    That is, $\exists\alpha_1,\dots,\alpha_n$ such that $\sum_{i=1}^{n}\alpha_iT(\mathbf{v}_i)=\mathbf{0}_W$.

    \begin{equation}
        \begin{aligned}
            \sum_{i=1}^{n}\alpha_iT(\mathbf{v}_i)&=\mathbf{0}_W\\
            T\left(\sum_{i=1}^{n}\alpha_i\mathbf{v}_i\right)&=\mathbf{0}_W
        \end{aligned}
    \end{equation}

    $\sum_{i=1}^{n}\alpha_i\mathbf{v}_i\not=\mathbf{0}_V$. But $\textrm{Ker}(T)=\{\mathbf{0}_V\}$. Contradiction!

    $\therefore$

    \begin{equation}
        \{T(\mathbf{v}_1),\dots,T(\mathbf{v}_n)\}\ \textrm{is linearly independent}
    \end{equation}

    Hence, $\{T(\mathbf{v}_1),\dots,T(\mathbf{v}_n)\}$ is a basis of $W$.
\end{proof}

\pagebreak

\section{Question 4}

\begin{proof}{}
    \noindent

    Let $dim(V)=dim(W)=n$.

    Pick one basis respectively for $V$ and $W$:

    \[
        \mathcal{B}_V=\{\mathbf{v}_1,\dots,\mathbf{v}_n\},\ \mathcal{B}_W=\{\mathbf{w}_1,\dots,\mathbf{w}_n\}
    \]

    Define a linear transformation $T:V\rightarrow W$,

    First define $T:\mathcal{B}_V\rightarrow\mathcal{B}_W$:

    \begin{equation}
        \forall i=1,\dots,n,\ \mathbf{v}_i=\mathbf{w}_i
    \end{equation}

    Since

    \[
        \forall\mathbf{v}\in V,\exists\alpha_1,\dots,\alpha_n,\ \textrm{such that}\ \mathbf{v}=\sum_{i=1}^{n}\alpha_i\mathbf{v}_i
    \]

    So define

    \[
        T(\mathbf{v}):=\sum_{i=1}^{n}\alpha_iT(\mathbf{v}_i)=\sum_{i=1}^{n}\alpha_i\mathbf{w}_i
    \]

    $\because$

    \[
        \forall\mathbf{w}\in W,\exists\beta_1,\dots,\beta_n,\ \textrm{such that}\ \mathbf{w}=\sum_{i=1}^{n}\beta_i\mathbf{w}_i
    \]

    $\therefore$

    \[
        \forall\mathbf{w}\in W,\mathbf{w}=\sum_{i=1}^{n}\beta_iT(\mathbf{v}_i)=T(\sum_{i=1}^{n}\beta_i\mathbf{v}_i)
    \]that is, $T$ is surjective.

    Suppose $T$ is not injective. That is

    \[
        \exists\mathbf{v}\in V,\mathbf{v}\not=\mathbf{0}_V,T(\mathbf{v})=\mathbf{0}_W
    \]

    $\because\exists\alpha_1,\dots,\alpha_n$ not all zero, such that $\mathbf{v}=\sum_{i=1}^{n}\alpha_i\mathbf{v}_i$

    $\therefore$

    \[
        T(\mathbf{v})=\sum_{i=1}^{n}\alpha_i\mathbf{w}_i=\mathbf{0}_W
    \]that is, $\mathcal{B}_W$ is linearly dependent. Contradiction!

    $\therefore T$ must be injective.

    Hence, $T$ is bijective, which is an isomorphism.
\end{proof}

\pagebreak

\section{Question 5}
\noindent

$\because (1,3,0,0,0)$ and $(0,0,1,1,1)$ are independent

and $Span(\{(1,3,0,0,0),(0,0,1,1,1)\})=\textrm{Nul}(T)$

$\therefore dim(\textrm{Nul}(T))=2$

So

$dim(\textrm{Nul}(T))+dim(\textrm{Image}(T))=2+dim(\textrm{Image}(T))\leq2+dim(\mathbf{F}^2)=4$

However, $dim(\textrm{Nul}(T))+dim(\textrm{Image}(T))=dim(\mathbb{F}^5)=5$. Contradiction!

Hence, there is no such a linear transformation.

\pagebreak

\section{Question 6}

\begin{proof}{}
    \noindent

    \noindent
    \textbf{(a)}

    $\because\forall x\in\mathbb{F},y\in\mathbb{F},\mathbf{v}_1\in V,\mathbf{v}_2\in V,$

    \[
        \begin{aligned}
            \alpha T(x\mathbf{v}_1+y\mathbf{v}_2)&=\alpha (T(x\mathbf{v}_1+y\mathbf{v}_2))\\
            &=\alpha (xT(\mathbf{v}_1)+yT(\mathbf{v}_2))\\
            &=x(\alpha T)(\mathbf{v}_1)+y(\alpha T)(\mathbf{v}_2)\in W
        \end{aligned}
    \]

    $\therefore \alpha T\in Hom_{\mathbb{F}}(V,W)$.

    $\because\forall x\in\mathbb{F},y\in\mathbb{F},\mathbf{v}_1\in V,\mathbf{v}_2\in V,$

    \[
        \begin{aligned}
            (T+S)(x\mathbf{v}_1+y\mathbf{v}_2)&=T(x\mathbf{v}_1+y\mathbf{v}_2)+S(x\mathbf{v}_1+y\mathbf{v}_2)\\
            &=xT(\mathbf{v}_1)+yT(\mathbf{v}_2)+xS(\mathbf{v}_1)+yS(\mathbf{v}_2)\\
            &=x(T+S)(\mathbf{v}_1)+y(T+S)(\mathbf{v}_2)\in W
        \end{aligned}
    \]

    $\therefore (T+S)\in Hom_{\mathbb{F}}(V,W)$.

    \noindent
    \textbf{(b)}

    Define $T_0(\mathbf{v})=\mathbf{0}_W,\forall\mathbf{v}\in V$.

    $\because \forall T\in Hom_{\mathbb{F}}(V,W),\forall\mathbf{v}\in V,(T+T_0)(\mathbf{v})=T(\mathbf{v})$
    
    that is $(T+T_0)=T$

    $\therefore$the additive identity in $Hom_{\mathbb{F}}(V,W)$ is $T_0$.
\end{proof}

\end{document}